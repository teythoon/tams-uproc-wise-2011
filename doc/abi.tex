\section{Application binary interface}

\subsection{Calling convention}

\newcommand{\I}[1]{\emph{#1}}

Function calls are implemented using the synthetic \I{call}
instruction. \I{call Reg0} is equivalent to \I{br \%lr, Reg0, \_,
  0}. \I{br} stores the return address \(\%pc + 4\) into \I{\%lr}.

The registers \(\%0 - \%7\) are used to hand parameters to the callee,
serve as scratch registers for the callee and can be used to return
values to the caller.

The caller must consider those registers dirty after a function call.
The callee can choose to hand results back to the caller.

TODO: push more arguments to the stack

\subsection{Function prologue}

\begin{lstlisting}
function_start:    push %pl, %lr, %bp, %st
                   mov %bp, %sp
\end{lstlisting}

\subsection{Function epilogue}

\begin{lstlisting}
                   mov %sp, %bp
                   pop %sp, %bp, %lr, %pl
                   ret
\end{lstlisting}
